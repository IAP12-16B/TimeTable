\RequirePackage[l2tabu, orthodox]{nag}
\documentclass[11pt,a4paper,titlepage,portrait,ngerman,final]{scrartcl}
\pdfgentounicode=1
\pdfcompresslevel=8

\RequirePackage[
	pdftex,
	plainpages=true,
	breaklinks=true,
	colorlinks=false,
	linktoc=all,
	linkcolor=false,
	pageanchor=true,
	unicode=true,
	pdfborder={0 0 0},
	hidelinks=true,
	pdfpagetransition=Dissolve,
	pdfdisplaydoctitle=true,
	final=true
]{hyperref}


% INFO SECTION %
\def \docTitle{Stundenplan}
\def \docSubtitle{M133}
\def \docAuthor{\href{http://kije.ch}{Kim Jeker}}
\def \docDate{\today}

\hypersetup{
	pdftitle=\docTitle,
	pdfsubject=\docSubtitle,
	pdfauthor=\docAuthor,
}

% PACKAGES
%\usepackage[osf]{mathpazo} 
\usepackage[default]{raleway}% Font
\usepackage[pdftex]{graphicx}
\usepackage{hfoldsty}
\usepackage{thumbpdf}
\usepackage{amsmath}
\usepackage[ngerman]{babel}
\usepackage{siunitx}
\usepackage{cleveref}
\usepackage{fancyhdr}
\usepackage[utf8]{inputenc}
\usepackage[T1]{fontenc}
\usepackage{ngerman}
\usepackage[a4paper]{geometry}
\usepackage[
	activate=true,
	babel=true,
	kerning=all, 
	tracking=all,
	letterspace=90
]{microtype}
\usepackage{fixltx2e}
\usepackage{currfile}
\usepackage{caption}
\usepackage{xcolor}
\usepackage{lastpage}
\usepackage{ellipsis}
\usepackage{bigfoot}
\usepackage{lipsum}

\setkomafont{disposition}{\normalfont}
\setkomafont{title}{\normalfont\bfseries\scshape\Huge}
\setkomafont{subtitle}{\normalfont\scshape\bfseries\normalsize}
\setkomafont{dictumauthor}{\normalfont\bfseries\small}
\setkomafont{section}{\newpage\normalfont\scshape\LARGE}
\setkomafont{subsection}{\normalfont\bfseries\scshape\large}
\setkomafont{subsubsection}{\normalfont\bfseries\normalsize}
\setkomafont{descriptionlabel}{\normalfont\bfseries\small}



% DOC INFO 
\title{\docTitle}
\subtitle{\docSubtitle}
\author{\large\scshape\docAuthor}
\date{\normalsize\scshape\docDate}

% HEADER & FOOTER
\pagestyle{fancy}
\lhead{\docTitle}
\chead{\textbf{\leftmark}}
\rhead{\docSubtitle}
\lfoot{\currfilename}
\cfoot{\textbf{\docAuthor}}
\rfoot{Seite \thepage\ von \pageref*{LastPage}}


% Make German ;)
\renewcommand{\figurename}{Abbildung}

\begin{document}
\maketitle
\tableofcontents
\newpage
\section{Konzept}
\subsection{Um was geht es?}
Mittels \textbf{JavaScript}, \textbf{HTML}, \textbf{CSS} und dem Framework \textbf{Bootstrap} wird eine Stundenplan-Applikation umgesetzt, welche die Stundenpläne der GIBM anzeigt.

\subsection{Ursprung der Daten}
Die Daten werden von einem externen PHP-Script, welches auf einem Server der GIBM läuft, bereitgestellt.
\subsubsection*{Interfaces:}
\begin{description}
	\item[Berufe]{\url{http://home.gibm.ch/interfaces/133/berufe.php}}
	\item[Klassen]{\url{http://home.gibm.ch/interfaces/133/klassen.php}}
	\item[Stundentafel]{\url{http://home.gibm.ch/interfaces/133/tafel.php}}
\end{description}

\subsection{Format der Daten}
Die Daten werden im \textbf{JSON}-Format zurückgeiefert.
\subsection{Umgesetzte Zusatzfunktionen}
\begin{itemize}
\item Klassenwahl in Cookie speichern
\item Transitions
\item Applikation ist Mobile-Tauglich
\end{itemize}
\section{Testplan}
\vspace{2.5mm}
\noindent
\begin{tabular}{|p{0.3\textwidth}||p{0.63\textwidth}|}
\hline \rule[-2ex]{0pt}{5.5ex} \textbf{Vorbedingung} & {
	\begin{list}{\textendash}{\vspace{-5mm}}
		\item{Webseite ist geöffnet \& Internetverbinung ist vorhanden}
	\end{list}
} \\ 
\hline \rule[-2ex]{0pt}{5.5ex} \textbf{Aktion} & {
	\begin{list}{\textendash}{\vspace{-5mm}}
		\item{Der User wählt den Beruf aus.}
	\end{list}
} \\ 
\hline \rule[-2ex]{0pt}{5.5ex} \textbf{Nachbedingung} & {
	\begin{list}{\textendash}{\vspace{-5mm}}
		\item{Die Liste mit Klassen hat sich aktualisiert. Nun sind nur noch alle Klassen, welche dem ausgewählten Beruf entspricht, auswählbar.}
	\end{list}
}  \\
\hline \rule[-2ex]{0pt}{5.5ex} \textbf{Aktion} & {
	\begin{list}{\textendash}{\vspace{-5mm}}
		\item{Der User wählt eine Klasse aus}
	\end{list}
} \\ 
\hline \rule[-2ex]{0pt}{5.5ex} \textbf{Nachbedingung} & {
	\begin{list}{\textendash}{\vspace{-5mm}}
		\item{Der Stundenplan hat sich aktualisiert. Es sind nun die Stunden der aktuellen Klassen sichtbar.}
	\end{list}
}  \\
\hline \rule[-2ex]{0pt}{5.5ex} \textbf{Aktion} & {
	\begin{list}{\textendash}{\vspace{-5mm}}
		\item{Der User schliesst den Browser und öffnet Ihn erneut.}
	\end{list}
} \\ 
\hline \rule[-2ex]{0pt}{5.5ex} \textbf{Nachbedingung} & {
	\begin{list}{\textendash}{\vspace{-5mm}}
		\item{Nach dem Öffnen sollten wieder der gleiche Beruf und die gleiche Klasse ausgewählt, wie vor dem schliessen.}
	\end{list}
}  \\
\hline \rule[-2ex]{0pt}{5.5ex} \textbf{Ausnahme} & {
	\begin{list}{\textendash}{\vspace{-5mm}}
		\item{
			\textbf{Keine Internetverbindung}
			
			Der User erhält eine Meldung.
		}
		\item{
			\textbf{User hat ein Add-On, welches Cookies beim schliessen des Browsers löscht.}
			
			Ausgewählter Beruf \& Klasse werden nicht gespeichert.
		}
	\end{list}
}  \\
\hline 
\end{tabular}
\end{document}
